\chapter{Introduction}

% Le Monde et ses thèmes

% Système de jeu


\DndDropCapLine{L}{'espèce humaine est sous la domination} des maîtres depuis aussi loin que les récits remontent.
Nous sommes leurs forces de travail dans les champs ou les mines. 
Nous sommes leurs fidèles compagnons qui gardent leurs demeures.
Nous sommes les chiens de leurs armées.
Nous sommes leurs rats de laboratoires. 
Nous sommes la viande qui garnie leurs tables.
Nous sommes les clowns qui les divertissent.
Nous sommes la vermine qui infestent leurs rues.
Ils sont nos \emph{Maîtres} et bien plus que cela. Ils sont nos \emph{Dieux}.

\section{Le Monde}
Depuis des âges immémoriaux, l'humanité est aux mains d'une race d'êtres reptiliens: les \emph{Maîtres}. Ceux-ci domestiquent les humains pour leurs besoins donnant lieux à plusieurs espèces humaines. La peur, la docilité ou la fidélité qui leur sont vouées les ont jusqu'ici préservé d'une quelconque révolte.

\paragraph{}
De par leurs expériences chimiques et biologiques, les poisons de leurs industries et les guerres incessantes qu'ils mènent, la pollution est présente sous toutes ses formes. L'air est à peine respirable et il est fréquent de ne pas voir à 10 mètres dans le brouillard permanent des villes. Les odeurs sulfuriques et acides des laboratoires et des usines écorchent les poumons et les yeux. L'eau mêlée de solvants et de restes organiques ne peut être bue sans haut-le-coeur. La terre est pleine des oxydes des métaux des villes, des débris jetés, des traces des guerres.

Cette pollution ambiante provoque l'instabilité biologique de nombreuses espèces. Des mutations affectent une grande partie des créatures, même si les Maîtres les plus puissants ne sont pas touchés. De nombreuses expériences sur ces mutations sont entreprises par les Maîtres afin de bénéficier de mutations à leur avantage. C'est au sein de leurs laboratoires que les mutations les plus importantes ont lieues. Ces expériences ne donne pas d'importance au nombre de vies prises pour créer des drogues, forcer des mutations, transplanter les organes entre plusieurs espèces, façonner des chimères en mélangeant des êtres vivants. Parmi les mutations étranges sorties de ces laboratoires, certaines ont permit la découverte d'une nouvelle énergie ainsi qu'un moyen de la manipuler: la magie.

\paragraph{}
Malgré la main mise des Maîtres sur les hommes, des rumeurs circulent. La révolte gronde. Vérité ou tromperie? Quoi qu'il en soit, tous sont concernés.

\section{Le Jeu}

\emph{Bétail} est un jeu de rôle plongeant les joueurs dans un monde post-apocalyptique où les êtres humains n'y sont que des animaux domestiqués par une race supérieure. La plupart des êtres humains ne sont pas maîtres de leur destin. Ils peuvent alors batailler pour atteindre une liberté à laquelle ils n'ont jamais eu droit, chercher à survivre cet environnement dangereux et sans pitié, obtenir la reconnaissance de leur maître par leurs actions ou encore chercher à déjouer les ruses et les tromperies des Maîtres pour guider l'humanité vers un avenir meilleur. Pour ce faire, ils ne peuvent se fier qu'à eux-même, à la justesse de leurs actions ainsi qu'à leur chance.

Pour ce faire, les joueurs basent la réussite de leurs actions sur un jet de dé. Selon les spécificité des personnages, les chances sont biaisées du fait de leur maîtrise. En effet, les possibilités des personnages sont regroupés selon $6$ catégories:
\begin{itemize}
    \item Le \textbf{Savoir} regroupe la connaissance que possède le personnage, sa capacité à analyser un document, un objet ou à savoir comment est construit quelque chose.
    \item La \textbf{Perspicacité} est la capacité à observer son environnement et en tirer avantage.
    \item Le \textbf{Bagout} représente la facilité à s'exprimer, à savoir jouer la comédie ou même mentir.
    \item L'\textbf{Agilité} est la capacité à se mouvoir, bouger son corps de manière contrôlée pour escalader, se faufiler ou porter un coup.
    \item La \textbf{Dextérité} regroupe les compétences pour effectuer des actions précises et nécessitant un doigté particulier comme forcer une serrure, créer un objet ou amorcer/désamorcer un piège.
    \item La \textbf{Constitution} représente la puissance physique, la résistance à la douleur et aux maladies ou encore l'endurance du personnage.
\end{itemize}

La valeur de ces caractéristiques indiquent les bonus et malus de maîtrise. Les niveaux de maîtrise sont les suivants:
\begin{itemize}
    \item 1 signifie un gros désavantage. Il faut alors lancer 3D10 et garder le résultat le plus bas.
    \item 2 signifie un désavantage simple. Il faut alors lancer 2D10 et garder le résultat le plus bas.
    \item 3 signifie que la personne possède une légère maîtrise de la caractéristique. Il faut alors 1D10 qui indique le résultat.
    \item 4 et + signifie un avantage de plus en plus important. Pour une valeur de 4, il faut lancer 2D10 et garder le meilleur résultat. Pour chaque niveau supérieur à 4, 1D10 est ajouté au jet et le meilleur résultat est gardé.
\end{itemize}

Au résultat du jet de caractéristique est ajouté un score de compétence. 
Si le résultat du dé est de 0 alors il s'agit d'un échec critique et l'action a échoué de manière spectaculaire ce qui entraîne souvent une conséquence désastreuse pour les joueurs à moins que vous ne préfériez l'utiliser en ressort comique.  Au contraire, un 9 indique un succès critique et l'action a réussi de façon extraordinaire même si celle-ci semblait à première vue impossible ce qui met les joueurs en position d'avantage.

\section{L'Aventure}

S'infiltrer au coeur des laboratoires des Maîtres, poursuivre les chiens de la Révolution, se battre pour la liberté ou encore obtenir la gloire dans les arènes des villes. Autant de possibilités de vivre des aventures et de poursuivre suivre ses idéaux dans un monde brutal et malade qui laisse peu de place à la faiblesse.

L'aventure que va vivre les joueurs est souvent écrite par vous, le Tisseur, ou reprise d'une aventure déjà existante. Toujours est-il qu'elle se passe rarement comme prévue et le Monde évolue en fonction des actions des personnages. Même les plus insignifiantes des décisions peut mener à une modification majeure de la trame de l'univers. C'est cette évolution et cette multitude de choix possibles qui fait la richesse d'un univers alors quand l'aventure dérive, improvisez et laissez les joueurs vous mener. Après tout, se sont eux, les vrais Maîtres de leur aventure.

\subsection{Les ficelles de l'aventure}

Quelle que soit l'aventure, elle inclut souvent les mêmes ficelles en proportions différentes.

\paragraph{L'exploration} qui correspond à la découverte d'un nouvel environnement, de nouveaux dangers auxquels ils faut s'adapter. La progression face à l'inconnu ainsi que la collecte d'information. En effet, il est mal avisé de s'aventurer dans les méandres d'un laboratoire ou d'un manoir Maître sans tout d'abord vérifier la présence ou non de ceux-ci, trouver une voie de repli ou au moins savoir ce qui s'y cache. Car malheur à ceux qui mal préparés se font attraper, un rat est rarement apprécié dans une maison.

\paragraph{Les interactions sociales} sont souvent une composante importante. En effet, il ne faut pas négliger la quantité d'information qu'un PNJ peut avoir sur un lieu, une créature ou encore les relations entre d'autres individus. Ces informations peuvent se révéler vitales afin de réussir à se faire bien voir des uns et échapper aux autres.

\paragraph{Les combats} à proprement parler sont en général rares et on lieu avec des armes bricolées sur le tas. Les affrontement se font souvent avec d'autres humains, des créatures mutantes ou un Maître affaiblit. En effet, les Maîtres sont très souvent bien mieux équipés, nourris et plus forts que vos joueurs et les Chimères possèdent des aptitudes inconnues et sont bien plus puissantes physiquement. La discrétion et la fuite sont donc bien souvent de mise.
Cependant, il est parfois inévitable. Une structure de l'action plus stricte est alors requise. Il se déroule effectivement dans un ordre des personnages précis relatif à leur initiative sur les autres, leur rapidité d'action. Cependant, il est également possible de définir ses actions par rapport à celle des autres, ou encore en réaction comme décider d'attaquer la personne dès qu'elle passe l'angle du mur. De plus, cet ordre peut varier selon les situations comme lorsqu'on prend quelqu'un par surprise ou qu'il est entravé. Il faut donc réussir à jouer de l'environnement à son avantage pour le placer en position de supériorité. 

Mais l'action en combat ne se résume pas à attaquer sans réfléchir. Un tour se constitue d'une action et d'un déplacement et à une durée temporelle de 5 secondes environ. L'action peut être une attaque, une réplique, une esquive ou tout autre action ue le joueur peut décider de faire dans la situation. Il est également possible d'effectuer plusieurs actions en même temps.

\subsection{Les mutations}

Dans l'univers de \emph{Bétail}, la pollution contamine souvent les êtres vivants. Cela les amènent souvent à contracter des hôtes parasitaires qui vivent en symbiose ou pas, ou alors des mutations. Ces mutations présentent souvent un désavantage pour les personnes les contractant mais apporte parfois de nouvelles capacités selon comment elles sont utilisées. 

Ces mutations sont souvent de cause naturelles et présentes dès la naissance et s'accentuent avec le temps. Cependant, de nombreuses expériences biologiques et chimiques sont effectuées sur des êtres vivants, provoquant ces mutations. Celles-ci sont souvent plus importantes que les naturelles et souvent mortelles pour les cobayes.

\subsection{La Magie}

Parmi toutes les mutations possibles, l'une d'entre elles permet notamment de percevoir une énergie présente dans l'environnement: la magie. Cette force est présente de manière inégale selon les lieux et influe sur les réactions qui s'y passent. Il est également possible de manipuler cette énergie pour déclencher des réactions à son gré. Cependant, cela requiert une seconde mutation particulière. 

\paragraph{Les types de magies}
Cette mutation peut prendre plusieurs formes et les plus connues se catégorisent en trois groupes. La magie Biologique agit sur le corps, permettant de modifier celui-ci. La magie Spirituelle permet de contrôler son esprit, ou celui des autres. La magie de Transmutation modifie les réactions chimiques environnantes, les accélérant ou les ralentissant voire même les déclencher. 